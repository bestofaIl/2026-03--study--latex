% Options for packages loaded elsewhere
% Options for packages loaded elsewhere
\PassOptionsToPackage{unicode}{hyperref}
\PassOptionsToPackage{hyphens}{url}
%
\documentclass[
  english,
  russian,
  12pt,
  a4paper,
  DIV=11,
  numbers=noendperiod]{scrreprt}
\usepackage{xcolor}
\usepackage{amsmath,amssymb}
\setcounter{secnumdepth}{5}
\usepackage{iftex}
\ifPDFTeX
  \usepackage[T1]{fontenc}
  \usepackage[utf8]{inputenc}
  \usepackage{textcomp} % provide euro and other symbols
\else % if luatex or xetex
  \usepackage{unicode-math} % this also loads fontspec
  \defaultfontfeatures{Scale=MatchLowercase}
  \defaultfontfeatures[\rmfamily]{Ligatures=TeX,Scale=1}
\fi
\usepackage{lmodern}
\ifPDFTeX\else
  % xetex/luatex font selection
\fi
% Use upquote if available, for straight quotes in verbatim environments
\IfFileExists{upquote.sty}{\usepackage{upquote}}{}
\IfFileExists{microtype.sty}{% use microtype if available
  \usepackage[]{microtype}
  \UseMicrotypeSet[protrusion]{basicmath} % disable protrusion for tt fonts
}{}
\usepackage{setspace}
% Make \paragraph and \subparagraph free-standing
\makeatletter
\ifx\paragraph\undefined\else
  \let\oldparagraph\paragraph
  \renewcommand{\paragraph}{
    \@ifstar
      \xxxParagraphStar
      \xxxParagraphNoStar
  }
  \newcommand{\xxxParagraphStar}[1]{\oldparagraph*{#1}\mbox{}}
  \newcommand{\xxxParagraphNoStar}[1]{\oldparagraph{#1}\mbox{}}
\fi
\ifx\subparagraph\undefined\else
  \let\oldsubparagraph\subparagraph
  \renewcommand{\subparagraph}{
    \@ifstar
      \xxxSubParagraphStar
      \xxxSubParagraphNoStar
  }
  \newcommand{\xxxSubParagraphStar}[1]{\oldsubparagraph*{#1}\mbox{}}
  \newcommand{\xxxSubParagraphNoStar}[1]{\oldsubparagraph{#1}\mbox{}}
\fi
\makeatother


\usepackage{longtable,booktabs,array}
\usepackage{calc} % for calculating minipage widths
% Correct order of tables after \paragraph or \subparagraph
\usepackage{etoolbox}
\makeatletter
\patchcmd\longtable{\par}{\if@noskipsec\mbox{}\fi\par}{}{}
\makeatother
% Allow footnotes in longtable head/foot
\IfFileExists{footnotehyper.sty}{\usepackage{footnotehyper}}{\usepackage{footnote}}
\makesavenoteenv{longtable}
\usepackage{graphicx}
\makeatletter
\newsavebox\pandoc@box
\newcommand*\pandocbounded[1]{% scales image to fit in text height/width
  \sbox\pandoc@box{#1}%
  \Gscale@div\@tempa{\textheight}{\dimexpr\ht\pandoc@box+\dp\pandoc@box\relax}%
  \Gscale@div\@tempb{\linewidth}{\wd\pandoc@box}%
  \ifdim\@tempb\p@<\@tempa\p@\let\@tempa\@tempb\fi% select the smaller of both
  \ifdim\@tempa\p@<\p@\scalebox{\@tempa}{\usebox\pandoc@box}%
  \else\usebox{\pandoc@box}%
  \fi%
}
% Set default figure placement to htbp
\def\fps@figure{htbp}
\makeatother



\ifLuaTeX
\usepackage[bidi=basic,provide=*]{babel}
\else
\usepackage[bidi=default,provide=*]{babel}
\fi
% get rid of language-specific shorthands (see #6817):
\let\LanguageShortHands\languageshorthands
\def\languageshorthands#1{}


\setlength{\emergencystretch}{3em} % prevent overfull lines

\providecommand{\tightlist}{%
  \setlength{\itemsep}{0pt}\setlength{\parskip}{0pt}}



 
\usepackage[backend=biber,langhook=extras,autolang=other*]{biblatex}
\addbibresource{bib/cite.bib}

\usepackage[]{csquotes}

\usepackage{indentfirst}
\usepackage{float}
\floatplacement{figure}{H}
\IfFileExists{plex-otf.sty}{
  %% Full TeXlive
  \IfPackageAtLeastTF{plex-otf.sty}{2024-12-06}{
    \usepackage[math,RM={Scale=0.94},SS={Scale=0.94},SScon={Scale=0.94},TT={Scale=MatchLowercase,FakeStretch=0.9},DefaultFeatures={Ligatures=Common}]{plex-otf}
  }{
    \usepackage[RM={Scale=0.94},SS={Scale=0.94},SScon={Scale=0.94},TT={Scale=MatchLowercase,FakeStretch=0.9},DefaultFeatures={Ligatures=Common}]{plex-otf}
  }
}{
  %% TinyTeX
  \usepackage{libertine}
}
\KOMAoption{captions}{tableheading}
\makeatletter
\@ifpackageloaded{caption}{}{\usepackage{caption}}
\AtBeginDocument{%
\ifdefined\contentsname
  \renewcommand*\contentsname{Содержание}
\else
  \newcommand\contentsname{Содержание}
\fi
\ifdefined\listfigurename
  \renewcommand*\listfigurename{Список иллюстраций}
\else
  \newcommand\listfigurename{Список иллюстраций}
\fi
\ifdefined\listtablename
  \renewcommand*\listtablename{Список таблиц}
\else
  \newcommand\listtablename{Список таблиц}
\fi
\ifdefined\figurename
  \renewcommand*\figurename{Рисунок}
\else
  \newcommand\figurename{Рисунок}
\fi
\ifdefined\tablename
  \renewcommand*\tablename{Таблица}
\else
  \newcommand\tablename{Таблица}
\fi
}
\@ifpackageloaded{float}{}{\usepackage{float}}
\floatstyle{ruled}
\@ifundefined{c@chapter}{\newfloat{codelisting}{h}{lop}}{\newfloat{codelisting}{h}{lop}[chapter]}
\floatname{codelisting}{Список}
\newcommand*\listoflistings{\listof{codelisting}{Листинги}}
\makeatother
\makeatletter
\makeatother
\makeatletter
\@ifpackageloaded{caption}{}{\usepackage{caption}}
\@ifpackageloaded{subcaption}{}{\usepackage{subcaption}}
\makeatother
\usepackage{bookmark}
\IfFileExists{xurl.sty}{\usepackage{xurl}}{} % add URL line breaks if available
\urlstyle{same}
\hypersetup{
  pdftitle={Лабораторная работа №4},
  pdfauthor={Абрамян Артём Арменович},
  pdflang={ru-RU},
  hidelinks,
  pdfcreator={LaTeX via pandoc}}


\title{Лабораторная работа №4}
\usepackage{etoolbox}
\makeatletter
\providecommand{\subtitle}[1]{% add subtitle to \maketitle
  \apptocmd{\@title}{\par {\large #1 \par}}{}{}
}
\makeatother
\subtitle{Работа с графикой в LaTeX}
\author{Абрамян Артём Арменович}
\date{}
\begin{document}
\maketitle

\renewcommand*\contentsname{Содержание}
{
\setcounter{tocdepth}{1}
\tableofcontents
}
\listoffigures
\listoftables

\setstretch{1.5}
\chapter{Цель
работы}\label{ux446ux435ux43bux44c-ux440ux430ux431ux43eux442ux44b}

Цель данной работы --- изучить возможности включения и манипулирования
графикой в LaTeX, освоить различные способы позиционирования
изображений, научиться работать с перекрестными ссылками.

\chapter{Задание}\label{ux437ux430ux434ux430ux43dux438ux435}

Выполнить следующие задания:

\begin{itemize}
\tightlist
\item
  Попробовать включить собственное изображение, заменив стандартные
  примеры.
\item
  Исследовать возможности параметров \texttt{height}, \texttt{width},
  \texttt{angle} и \texttt{scale}.
\item
  Использовать параметр \texttt{width} для установки размера графики
  относительно \texttt{textwidth} и другой графики относительно
  \texttt{linewidth}. Протестировать их поведение с опцией
  \texttt{twocolumn} и без нее.
\item
  Создать демонстрацию с использованием \texttt{lipsum} и попробовать
  различные спецификаторы позиционирования флоатов. Изучить, как
  различные спецификаторы взаимодействуют друг с другом.
\item
  Добавить новые нумерованные части (разделы, подразделы, нумерованные
  списки) в тестовый документ и определить, сколько компиляций
  необходимо для корректной работы команд \texttt{label}.
\item
  Добавить флоаты и проверить, что происходит при размещении
  \texttt{label} до \texttt{caption} вместо после; предсказать
  результат.
\item
  Выяснить, что происходит при размещении \texttt{label} для уравнения
  после \texttt{end\ equation}.
\end{itemize}

\chapter{Теоретическое
введение}\label{ux442ux435ux43eux440ux435ux442ux438ux447ux435ux441ux43aux43eux435-ux432ux432ux435ux434ux435ux43dux438ux435}

Для включения графики в LaTeX используется пакет \texttt{graphicx},
который добавляет команду \texttt{includegraphics}. Этот пакет
поддерживает различные форматы изображений: EPS, PNG, JPG и PDF.

Изображения в LaTeX обычно включаются как плавающие объекты (floats),
чтобы избежать больших пустых промежутков на странице. Для управления
позиционированием используются спецификаторы:

\begin{itemize}
\tightlist
\item
  \texttt{h} --- \enquote*{Here} (здесь, если возможно)
\item
  \texttt{t} --- Top (верх страницы)
\item
  \texttt{b} --- Bottom (низ страницы)
\item
  \texttt{p} --- Page (отдельная страница только для флоатов)
\item
  \texttt{H} --- абсолютное позиционирование (требует пакет
  \texttt{float})
\end{itemize}

Система перекрестных ссылок в LaTeX работает с помощью команд
\texttt{label} и \texttt{ref}. Для корректной работы требуется минимум
две компиляции документа.

Более подробно про работу с графикой в LaTeX см. в
\autocite{kotelnikov_chebotaev_book_latex2_ru,lvovsky_book_latex_ru}.

\chapter{Выполнение лабораторной
работы}\label{ux432ux44bux43fux43eux43bux43dux435ux43dux438ux435-ux43bux430ux431ux43eux440ux430ux442ux43eux440ux43dux43eux439-ux440ux430ux431ux43eux442ux44b}

\section{Базовое включение
графики}\label{ux431ux430ux437ux43eux432ux43eux435-ux432ux43aux43bux44eux447ux435ux43dux438ux435-ux433ux440ux430ux444ux438ux43aux438}

Начнем работу с базового включения изображений в документ. Для этого
используется пакет \texttt{graphicx} и команда \texttt{includegraphics}.
На рисунке \ref{fig-001} показан простейший пример включения изображения
в документ.

\begin{figure}

\centering{

\includegraphics[width=0.7\linewidth,height=\textheight,keepaspectratio]{image/fig01.png}

}

\caption{\label{fig-001}Базовый пример включения изображения}

\end{figure}%

\section{Манипулирование размерами
графики}\label{ux43cux430ux43dux438ux43fux443ux43bux438ux440ux43eux432ux430ux43dux438ux435-ux440ux430ux437ux43cux435ux440ux430ux43cux438-ux433ux440ux430ux444ux438ux43aux438}

\subsection{Параметр
width}\label{ux43fux430ux440ux430ux43cux435ux442ux440-width}

Параметр \texttt{width} позволяет задать ширину изображения. Можно
использовать как абсолютные значения (например, 5cm), так и
относительные (например, 0.5 от ширины текста). На рисунке \ref{fig-002}
показан пример использования параметра \texttt{width}.

\begin{figure}

\centering{

\includegraphics[width=0.7\linewidth,height=\textheight,keepaspectratio]{image/fig02.png}

}

\caption{\label{fig-002}Изображение с параметром width}

\end{figure}%

\subsection{Параметр
height}\label{ux43fux430ux440ux430ux43cux435ux442ux440-height}

Аналогично ширине, можно задать высоту изображения с помощью параметра
\texttt{height}. LaTeX автоматически сохраняет пропорции изображения
(рис. \ref{fig-003}).

\begin{figure}

\centering{

\includegraphics[width=0.7\linewidth,height=\textheight,keepaspectratio]{image/fig03.png}

}

\caption{\label{fig-003}Изображение с параметром height}

\end{figure}%

\subsection{Параметр
scale}\label{ux43fux430ux440ux430ux43cux435ux442ux440-scale}

Параметр \texttt{scale} позволяет масштабировать изображение
пропорционально. Например, \texttt{scale=0.5} уменьшит изображение в два
раза (рис. \ref{fig-004}).

\begin{figure}

\centering{

\includegraphics[width=0.7\linewidth,height=\textheight,keepaspectratio]{image/fig04.png}

}

\caption{\label{fig-004}Использование параметра scale}

\end{figure}%

\subsection{Поворот
изображения}\label{ux43fux43eux432ux43eux440ux43eux442-ux438ux437ux43eux431ux440ux430ux436ux435ux43dux438ux44f}

С помощью параметра \texttt{angle} можно повернуть изображение на
заданный угол в градусах. На рисунке \ref{fig-005} показан пример
поворота изображения на 45 градусов.

\begin{figure}

\centering{

\includegraphics[width=0.7\linewidth,height=\textheight,keepaspectratio]{image/fig05.png}

}

\caption{\label{fig-005}Поворот изображения с помощью angle}

\end{figure}%

\section{Разница между textwidth и
linewidth}\label{ux440ux430ux437ux43dux438ux446ux430-ux43cux435ux436ux434ux443-textwidth-ux438-linewidth}

Параметр \texttt{textwidth} --- это ширина текстового блока на
физической странице, тогда как \texttt{linewidth} --- это текущая
ширина, которая может локально отличаться. На рисунке \ref{fig-006}
показан пример использования \texttt{textwidth}.

\begin{figure}

\centering{

\includegraphics[width=0.7\linewidth,height=\textheight,keepaspectratio]{image/fig06.png}

}

\caption{\label{fig-006}Пример с textwidth}

\end{figure}%

Разница наиболее заметна при использовании опции класса
\texttt{twocolumn}. В режиме двух колонок \texttt{linewidth}
адаптируется к ширине колонки, а \texttt{textwidth} всегда относится к
полной ширине страницы (рис. \ref{fig-007}).

\begin{figure}

\centering{

\includegraphics[width=0.7\linewidth,height=\textheight,keepaspectratio]{image/fig07.png}

}

\caption{\label{fig-007}Демонстрация разницы в режиме twocolumn}

\end{figure}%

В обычном однокалоночном режиме эти значения одинаковы, но в
\texttt{twocolumn} изображение с \texttt{width=0.5linewidth} будет
занимать половину ширины колонки, а с \texttt{width=0.5textwidth} ---
половину всей страницы.

\section{Позиционирование
флоатов}\label{ux43fux43eux437ux438ux446ux438ux43eux43dux438ux440ux43eux432ux430ux43dux438ux435-ux444ux43bux43eux430ux442ux43eux432}

\subsection{Спецификаторы
позиции}\label{ux441ux43fux435ux446ux438ux444ux438ux43aux430ux442ux43eux440ux44b-ux43fux43eux437ux438ux446ux438ux438}

Можно использовать различные спецификаторы позиции для управления
размещением изображений. На рисунке \ref{fig-008} показан пример кода с
различными спецификаторами.

\begin{figure}

\centering{

\includegraphics[width=0.7\linewidth,height=\textheight,keepaspectratio]{image/fig08.png}

}

\caption{\label{fig-008}Пример с различными спецификаторами позиции}

\end{figure}%

\subsection{Использование пакета
float}\label{ux438ux441ux43fux43eux43bux44cux437ux43eux432ux430ux43dux438ux435-ux43fux430ux43aux435ux442ux430-float}

Пакет \texttt{float} добавляет спецификатор \texttt{H}, который
размещает изображение точно в месте его определения в коде. Это показано
на рисунке \ref{fig-009}.

\begin{figure}

\centering{

\includegraphics[width=0.7\linewidth,height=\textheight,keepaspectratio]{image/fig09.png}

}

\caption{\label{fig-009}Использование спецификатора H}

\end{figure}%

Результат компиляции с различными флоатами можно увидеть на рисунке
\ref{fig-010}.

\begin{figure}

\centering{

\includegraphics[width=0.7\linewidth,height=\textheight,keepaspectratio]{image/fig10.png}

}

\caption{\label{fig-010}Результат компиляции с флоатами}

\end{figure}%

\section{Перекрестные
ссылки}\label{ux43fux435ux440ux435ux43aux440ux435ux441ux442ux43dux44bux435-ux441ux441ux44bux43bux43aux438}

\subsection{Механизм label и
ref}\label{ux43cux435ux445ux430ux43dux438ux437ux43c-label-ux438-ref}

Для автоматической нумерации элементов используется механизм меток. На
рисунке \ref{fig-011} показаны примеры использования \texttt{label} и
\texttt{ref}.

\begin{figure}

\centering{

\includegraphics[width=0.7\linewidth,height=\textheight,keepaspectratio]{image/fig11.png}

}

\caption{\label{fig-011}Примеры label и ref}

\end{figure}%

\subsection{Правильное размещение
label}\label{ux43fux440ux430ux432ux438ux43bux44cux43dux43eux435-ux440ux430ux437ux43cux435ux449ux435ux43dux438ux435-label}

Команда \texttt{label} всегда ссылается на предыдущую нумерованную
сущность. Для флоатов \texttt{label} должна идти после команды
\texttt{caption}. Правильный пример показан на рисунке \ref{fig-012}.

\begin{figure}

\centering{

\includegraphics[width=0.7\linewidth,height=\textheight,keepaspectratio]{image/fig12.png}

}

\caption{\label{fig-012}Правильное размещение метки}

\end{figure}%

\subsection{Неправильное размещение
метки}\label{ux43dux435ux43fux440ux430ux432ux438ux43bux44cux43dux43eux435-ux440ux430ux437ux43cux435ux449ux435ux43dux438ux435-ux43cux435ux442ux43aux438}

Если поместить \texttt{label} перед \texttt{caption}, метка будет
ссылаться на номер секции, а не на номер рисунка. Это демонстрируется на
рисунке \ref{fig-013}.

\begin{figure}

\centering{

\includegraphics[width=0.7\linewidth,height=\textheight,keepaspectratio]{image/fig13.png}

}

\caption{\label{fig-013}Неправильное размещение метки}

\end{figure}%

При неправильном размещении ссылка указывает на неверный номер.

\subsection{Метки для
уравнений}\label{ux43cux435ux442ux43aux438-ux434ux43bux44f-ux443ux440ux430ux432ux43dux435ux43dux438ux439}

Для уравнений метка должна находиться внутри окружения
\texttt{equation}, до команды закрывающей окружение. На рисунке
\ref{fig-014} показаны правильный и неправильный примеры.

\begin{figure}

\centering{

\includegraphics[width=0.7\linewidth,height=\textheight,keepaspectratio]{image/fig14.png}

}

\caption{\label{fig-014}Метки для уравнений}

\end{figure}%

Если метка размещена после закрытия окружения \texttt{equation}, она
будет ссылаться на номер раздела, а не на номер уравнения.

\subsection{Количество
компиляций}\label{ux43aux43eux43bux438ux447ux435ux441ux442ux432ux43e-ux43aux43eux43cux43fux438ux43bux44fux446ux438ux439}

Из-за использования вспомогательного файла первая компиляция документа
может показать вопросительные знаки вместо номеров ссылок (рис.
\ref{fig-015}).

\begin{figure}

\centering{

\includegraphics[width=0.7\linewidth,height=\textheight,keepaspectratio]{image/fig15.png}

}

\caption{\label{fig-015}Первая компиляция с вопросительными знаками}

\end{figure}%

После второй компиляции все ссылки становятся правильными (рис.
\ref{fig-016}).

\begin{figure}

\centering{

\includegraphics[width=0.7\linewidth,height=\textheight,keepaspectratio]{image/fig16.png}

}

\caption{\label{fig-016}Вторая компиляция с правильными номерами}

\end{figure}%

\section{Использование пакета
hyperref}\label{ux438ux441ux43fux43eux43bux44cux437ux43eux432ux430ux43dux438ux435-ux43fux430ux43aux435ux442ux430-hyperref}

Пакет \texttt{hyperref} позволяет превратить перекрестные ссылки в
гиперссылки. На рисунке \ref{fig-017} показан пример подключения пакета.

\begin{figure}

\centering{

\includegraphics[width=0.7\linewidth,height=\textheight,keepaspectratio]{image/fig17.png}

}

\caption{\label{fig-017}Подключение пакета hyperref}

\end{figure}%

С опцией \texttt{hidelinks} ссылки имеют тот же цвет, что и обычный
текст, но остаются кликабельными в PDF.

\chapter{Выводы}\label{ux432ux44bux432ux43eux434ux44b}

В ходе данной работы мы изучили продвинутые возможности работы с
графикой в LaTeX. Освоили:

\begin{itemize}
\tightlist
\item
  Базовое включение графики с помощью пакета \texttt{graphicx}
\item
  Манипулирование размерами изображений с помощью параметров
  \texttt{width}, \texttt{height}, \texttt{scale} и \texttt{angle}
\item
  Разницу между \texttt{textwidth} и \texttt{linewidth}, особенно в
  режиме \texttt{twocolumn}
\item
  Различные спецификаторы позиционирования флоатов (\texttt{h},
  \texttt{t}, \texttt{b}, \texttt{p}, \texttt{H})
\item
  Правильное использование системы перекрестных ссылок с помощью
  \texttt{label} и \texttt{ref}
\item
  Важность правильного размещения меток (после \texttt{caption} для
  флоатов, внутри окружения для уравнений)
\item
  Необходимость минимум двух компиляций для корректной работы
  перекрестных ссылок
\item
  Создание кликабельных гиперссылок с помощью пакета \texttt{hyperref}
\end{itemize}

Особое внимание было уделено распространенным ошибкам при размещении
меток и способам их избежания.

\chapter*{Список
литературы}\label{ux441ux43fux438ux441ux43eux43a-ux43bux438ux442ux435ux440ux430ux442ux443ux440ux44b}
\addcontentsline{toc}{chapter}{Список литературы}

\printbibliography[heading=none]





\end{document}
